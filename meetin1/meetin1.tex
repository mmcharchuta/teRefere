% Options for packages loaded elsewhere
\PassOptionsToPackage{unicode}{hyperref}
\PassOptionsToPackage{hyphens}{url}
%
\documentclass[
]{article}
\usepackage{amsmath,amssymb}
\usepackage{iftex}
\ifPDFTeX
  \usepackage[T1]{fontenc}
  \usepackage[utf8]{inputenc}
  \usepackage{textcomp} % provide euro and other symbols
\else % if luatex or xetex
  \usepackage{unicode-math} % this also loads fontspec
  \defaultfontfeatures{Scale=MatchLowercase}
  \defaultfontfeatures[\rmfamily]{Ligatures=TeX,Scale=1}
\fi
\usepackage{lmodern}
\ifPDFTeX\else
  % xetex/luatex font selection
\fi
% Use upquote if available, for straight quotes in verbatim environments
\IfFileExists{upquote.sty}{\usepackage{upquote}}{}
\IfFileExists{microtype.sty}{% use microtype if available
  \usepackage[]{microtype}
  \UseMicrotypeSet[protrusion]{basicmath} % disable protrusion for tt fonts
}{}
\makeatletter
\@ifundefined{KOMAClassName}{% if non-KOMA class
  \IfFileExists{parskip.sty}{%
    \usepackage{parskip}
  }{% else
    \setlength{\parindent}{0pt}
    \setlength{\parskip}{6pt plus 2pt minus 1pt}}
}{% if KOMA class
  \KOMAoptions{parskip=half}}
\makeatother
\usepackage{xcolor}
\usepackage[margin=1in]{geometry}
\usepackage{color}
\usepackage{fancyvrb}
\newcommand{\VerbBar}{|}
\newcommand{\VERB}{\Verb[commandchars=\\\{\}]}
\DefineVerbatimEnvironment{Highlighting}{Verbatim}{commandchars=\\\{\}}
% Add ',fontsize=\small' for more characters per line
\usepackage{framed}
\definecolor{shadecolor}{RGB}{248,248,248}
\newenvironment{Shaded}{\begin{snugshade}}{\end{snugshade}}
\newcommand{\AlertTok}[1]{\textcolor[rgb]{0.94,0.16,0.16}{#1}}
\newcommand{\AnnotationTok}[1]{\textcolor[rgb]{0.56,0.35,0.01}{\textbf{\textit{#1}}}}
\newcommand{\AttributeTok}[1]{\textcolor[rgb]{0.13,0.29,0.53}{#1}}
\newcommand{\BaseNTok}[1]{\textcolor[rgb]{0.00,0.00,0.81}{#1}}
\newcommand{\BuiltInTok}[1]{#1}
\newcommand{\CharTok}[1]{\textcolor[rgb]{0.31,0.60,0.02}{#1}}
\newcommand{\CommentTok}[1]{\textcolor[rgb]{0.56,0.35,0.01}{\textit{#1}}}
\newcommand{\CommentVarTok}[1]{\textcolor[rgb]{0.56,0.35,0.01}{\textbf{\textit{#1}}}}
\newcommand{\ConstantTok}[1]{\textcolor[rgb]{0.56,0.35,0.01}{#1}}
\newcommand{\ControlFlowTok}[1]{\textcolor[rgb]{0.13,0.29,0.53}{\textbf{#1}}}
\newcommand{\DataTypeTok}[1]{\textcolor[rgb]{0.13,0.29,0.53}{#1}}
\newcommand{\DecValTok}[1]{\textcolor[rgb]{0.00,0.00,0.81}{#1}}
\newcommand{\DocumentationTok}[1]{\textcolor[rgb]{0.56,0.35,0.01}{\textbf{\textit{#1}}}}
\newcommand{\ErrorTok}[1]{\textcolor[rgb]{0.64,0.00,0.00}{\textbf{#1}}}
\newcommand{\ExtensionTok}[1]{#1}
\newcommand{\FloatTok}[1]{\textcolor[rgb]{0.00,0.00,0.81}{#1}}
\newcommand{\FunctionTok}[1]{\textcolor[rgb]{0.13,0.29,0.53}{\textbf{#1}}}
\newcommand{\ImportTok}[1]{#1}
\newcommand{\InformationTok}[1]{\textcolor[rgb]{0.56,0.35,0.01}{\textbf{\textit{#1}}}}
\newcommand{\KeywordTok}[1]{\textcolor[rgb]{0.13,0.29,0.53}{\textbf{#1}}}
\newcommand{\NormalTok}[1]{#1}
\newcommand{\OperatorTok}[1]{\textcolor[rgb]{0.81,0.36,0.00}{\textbf{#1}}}
\newcommand{\OtherTok}[1]{\textcolor[rgb]{0.56,0.35,0.01}{#1}}
\newcommand{\PreprocessorTok}[1]{\textcolor[rgb]{0.56,0.35,0.01}{\textit{#1}}}
\newcommand{\RegionMarkerTok}[1]{#1}
\newcommand{\SpecialCharTok}[1]{\textcolor[rgb]{0.81,0.36,0.00}{\textbf{#1}}}
\newcommand{\SpecialStringTok}[1]{\textcolor[rgb]{0.31,0.60,0.02}{#1}}
\newcommand{\StringTok}[1]{\textcolor[rgb]{0.31,0.60,0.02}{#1}}
\newcommand{\VariableTok}[1]{\textcolor[rgb]{0.00,0.00,0.00}{#1}}
\newcommand{\VerbatimStringTok}[1]{\textcolor[rgb]{0.31,0.60,0.02}{#1}}
\newcommand{\WarningTok}[1]{\textcolor[rgb]{0.56,0.35,0.01}{\textbf{\textit{#1}}}}
\usepackage{graphicx}
\makeatletter
\def\maxwidth{\ifdim\Gin@nat@width>\linewidth\linewidth\else\Gin@nat@width\fi}
\def\maxheight{\ifdim\Gin@nat@height>\textheight\textheight\else\Gin@nat@height\fi}
\makeatother
% Scale images if necessary, so that they will not overflow the page
% margins by default, and it is still possible to overwrite the defaults
% using explicit options in \includegraphics[width, height, ...]{}
\setkeys{Gin}{width=\maxwidth,height=\maxheight,keepaspectratio}
% Set default figure placement to htbp
\makeatletter
\def\fps@figure{htbp}
\makeatother
\setlength{\emergencystretch}{3em} % prevent overfull lines
\providecommand{\tightlist}{%
  \setlength{\itemsep}{0pt}\setlength{\parskip}{0pt}}
\setcounter{secnumdepth}{-\maxdimen} % remove section numbering
\ifLuaTeX
  \usepackage{selnolig}  % disable illegal ligatures
\fi
\usepackage{bookmark}
\IfFileExists{xurl.sty}{\usepackage{xurl}}{} % add URL line breaks if available
\urlstyle{same}
\hypersetup{
  pdftitle={meetin 1},
  pdfauthor={mmcharchuta},
  hidelinks,
  pdfcreator={LaTeX via pandoc}}

\title{meetin 1}
\author{mmcharchuta}
\date{2025-04-01}

\begin{document}
\maketitle

\begin{Shaded}
\begin{Highlighting}[]
\CommentTok{\# Load the iris dataset}
\FunctionTok{data}\NormalTok{(iris)}

\CommentTok{\# Display the first few rows}
\FunctionTok{head}\NormalTok{(iris)}
\end{Highlighting}
\end{Shaded}

\begin{verbatim}
##   Sepal.Length Sepal.Width Petal.Length Petal.Width Species
## 1          5.1         3.5          1.4         0.2  setosa
## 2          4.9         3.0          1.4         0.2  setosa
## 3          4.7         3.2          1.3         0.2  setosa
## 4          4.6         3.1          1.5         0.2  setosa
## 5          5.0         3.6          1.4         0.2  setosa
## 6          5.4         3.9          1.7         0.4  setosa
\end{verbatim}

\begin{Shaded}
\begin{Highlighting}[]
\CommentTok{\# Check the structure of the dataset}
\FunctionTok{str}\NormalTok{(iris)}
\end{Highlighting}
\end{Shaded}

\begin{verbatim}
## 'data.frame':    150 obs. of  5 variables:
##  $ Sepal.Length: num  5.1 4.9 4.7 4.6 5 5.4 4.6 5 4.4 4.9 ...
##  $ Sepal.Width : num  3.5 3 3.2 3.1 3.6 3.9 3.4 3.4 2.9 3.1 ...
##  $ Petal.Length: num  1.4 1.4 1.3 1.5 1.4 1.7 1.4 1.5 1.4 1.5 ...
##  $ Petal.Width : num  0.2 0.2 0.2 0.2 0.2 0.4 0.3 0.2 0.2 0.1 ...
##  $ Species     : Factor w/ 3 levels "setosa","versicolor",..: 1 1 1 1 1 1 1 1 1 1 ...
\end{verbatim}

\begin{Shaded}
\begin{Highlighting}[]
\CommentTok{\# Summary statistics}
\FunctionTok{summary}\NormalTok{(iris)}
\end{Highlighting}
\end{Shaded}

\begin{verbatim}
##   Sepal.Length    Sepal.Width     Petal.Length    Petal.Width   
##  Min.   :4.300   Min.   :2.000   Min.   :1.000   Min.   :0.100  
##  1st Qu.:5.100   1st Qu.:2.800   1st Qu.:1.600   1st Qu.:0.300  
##  Median :5.800   Median :3.000   Median :4.350   Median :1.300  
##  Mean   :5.843   Mean   :3.057   Mean   :3.758   Mean   :1.199  
##  3rd Qu.:6.400   3rd Qu.:3.300   3rd Qu.:5.100   3rd Qu.:1.800  
##  Max.   :7.900   Max.   :4.400   Max.   :6.900   Max.   :2.500  
##        Species  
##  setosa    :50  
##  versicolor:50  
##  virginica :50  
##                 
##                 
## 
\end{verbatim}

\begin{Shaded}
\begin{Highlighting}[]
\CommentTok{\# Number of rows and columns}
\FunctionTok{cat}\NormalTok{(}\StringTok{"Number of rows:"}\NormalTok{, }\FunctionTok{nrow}\NormalTok{(iris), }\StringTok{"}\SpecialCharTok{\textbackslash{}n}\StringTok{"}\NormalTok{)}
\end{Highlighting}
\end{Shaded}

\begin{verbatim}
## Number of rows: 150
\end{verbatim}

\begin{Shaded}
\begin{Highlighting}[]
\FunctionTok{cat}\NormalTok{(}\StringTok{"Number of columns:"}\NormalTok{, }\FunctionTok{ncol}\NormalTok{(iris), }\StringTok{"}\SpecialCharTok{\textbackslash{}n}\StringTok{"}\NormalTok{)}
\end{Highlighting}
\end{Shaded}

\begin{verbatim}
## Number of columns: 5
\end{verbatim}

\begin{Shaded}
\begin{Highlighting}[]
\CommentTok{\# Check for missing values}
\FunctionTok{cat}\NormalTok{(}\StringTok{"Missing values:"}\NormalTok{, }\FunctionTok{sum}\NormalTok{(}\FunctionTok{is.na}\NormalTok{(iris)), }\StringTok{"}\SpecialCharTok{\textbackslash{}n}\StringTok{"}\NormalTok{)}
\end{Highlighting}
\end{Shaded}

\begin{verbatim}
## Missing values: 0
\end{verbatim}

\begin{Shaded}
\begin{Highlighting}[]
\CommentTok{\# Column data types}
\FunctionTok{sapply}\NormalTok{(iris, class)}
\end{Highlighting}
\end{Shaded}

\begin{verbatim}
## Sepal.Length  Sepal.Width Petal.Length  Petal.Width      Species 
##    "numeric"    "numeric"    "numeric"    "numeric"     "factor"
\end{verbatim}

\begin{Shaded}
\begin{Highlighting}[]
\CommentTok{\# Histogram for a numeric variable (e.g., Sepal.Length)}
\FunctionTok{hist}\NormalTok{(iris}\SpecialCharTok{$}\NormalTok{Sepal.Length, }\AttributeTok{main =} \StringTok{"Histogram of Sepal Length"}\NormalTok{, }\AttributeTok{xlab =} \StringTok{"Sepal Length"}\NormalTok{, }\AttributeTok{col =} \StringTok{"lightblue"}\NormalTok{)}
\end{Highlighting}
\end{Shaded}

\includegraphics{meetin1_files/figure-latex/visualizations-1.pdf}

\begin{Shaded}
\begin{Highlighting}[]
\CommentTok{\# Bar plot for a categorical variable (e.g., Species)}
\NormalTok{species\_counts }\OtherTok{\textless{}{-}} \FunctionTok{table}\NormalTok{(iris}\SpecialCharTok{$}\NormalTok{Species)}
\FunctionTok{barplot}\NormalTok{(species\_counts, }\AttributeTok{main =} \StringTok{"Bar Plot of Species"}\NormalTok{, }\AttributeTok{col =} \StringTok{"lightgreen"}\NormalTok{, }\AttributeTok{xlab =} \StringTok{"Species"}\NormalTok{, }\AttributeTok{ylab =} \StringTok{"Count"}\NormalTok{)}
\end{Highlighting}
\end{Shaded}

\includegraphics{meetin1_files/figure-latex/visualizations-2.pdf}

\begin{Shaded}
\begin{Highlighting}[]
\CommentTok{\# Most frequent category}
\FunctionTok{cat}\NormalTok{(}\StringTok{"Most frequent species:"}\NormalTok{, }\FunctionTok{names}\NormalTok{(}\FunctionTok{which.max}\NormalTok{(species\_counts)), }\StringTok{"}\SpecialCharTok{\textbackslash{}n}\StringTok{"}\NormalTok{)}
\end{Highlighting}
\end{Shaded}

\begin{verbatim}
## Most frequent species: setosa
\end{verbatim}

\begin{Shaded}
\begin{Highlighting}[]
\CommentTok{\# Density plot for the same numeric variable (e.g., Sepal.Length)}
\FunctionTok{plot}\NormalTok{(}\FunctionTok{density}\NormalTok{(iris}\SpecialCharTok{$}\NormalTok{Petal.Width), }\AttributeTok{main =} \StringTok{"Density Plot of Petal Width"}\NormalTok{, }
    \AttributeTok{xlab =} \StringTok{"Petal Width"}\NormalTok{, }\AttributeTok{col =} \StringTok{"darkblue"}\NormalTok{, }\AttributeTok{lwd =} \DecValTok{2}\NormalTok{)}

\CommentTok{\# Add a rug plot for individual data points}
\FunctionTok{rug}\NormalTok{(iris}\SpecialCharTok{$}\NormalTok{Sepal.Length, }\AttributeTok{col =} \StringTok{"red"}\NormalTok{)}
\end{Highlighting}
\end{Shaded}

\begin{verbatim}
## Warning in rug(iris$Sepal.Length, col = "red"): niektóre wartości będą
## przycięte
\end{verbatim}

\includegraphics{meetin1_files/figure-latex/visualizations-3.pdf}

\begin{Shaded}
\begin{Highlighting}[]
\CommentTok{\# Box plot for a numeric variable (e.g., Sepal.Width)}
\FunctionTok{boxplot}\NormalTok{(iris}\SpecialCharTok{$}\NormalTok{Sepal.Width, }\AttributeTok{main =} \StringTok{"Box Plot of Sepal Width"}\NormalTok{, }\AttributeTok{ylab =} \StringTok{"Sepal Width"}\NormalTok{, }\AttributeTok{col =} \StringTok{"orange"}\NormalTok{)}
\end{Highlighting}
\end{Shaded}

\includegraphics{meetin1_files/figure-latex/box_plots-1.pdf}

\begin{Shaded}
\begin{Highlighting}[]
\CommentTok{\# Add observations about median, quartiles, and outliers}
\FunctionTok{cat}\NormalTok{(}\StringTok{"Median:"}\NormalTok{, }\FunctionTok{median}\NormalTok{(iris}\SpecialCharTok{$}\NormalTok{Sepal.Width), }\StringTok{"}\SpecialCharTok{\textbackslash{}n}\StringTok{"}\NormalTok{)}
\end{Highlighting}
\end{Shaded}

\begin{verbatim}
## Median: 3
\end{verbatim}

\begin{Shaded}
\begin{Highlighting}[]
\FunctionTok{cat}\NormalTok{(}\StringTok{"1st Quartile:"}\NormalTok{, }\FunctionTok{quantile}\NormalTok{(iris}\SpecialCharTok{$}\NormalTok{Sepal.Width, }\FloatTok{0.25}\NormalTok{), }\StringTok{"}\SpecialCharTok{\textbackslash{}n}\StringTok{"}\NormalTok{)}
\end{Highlighting}
\end{Shaded}

\begin{verbatim}
## 1st Quartile: 2.8
\end{verbatim}

\begin{Shaded}
\begin{Highlighting}[]
\FunctionTok{cat}\NormalTok{(}\StringTok{"3rd Quartile:"}\NormalTok{, }\FunctionTok{quantile}\NormalTok{(iris}\SpecialCharTok{$}\NormalTok{Sepal.Width, }\FloatTok{0.75}\NormalTok{), }\StringTok{"}\SpecialCharTok{\textbackslash{}n}\StringTok{"}\NormalTok{)}
\end{Highlighting}
\end{Shaded}

\begin{verbatim}
## 3rd Quartile: 3.3
\end{verbatim}

\begin{Shaded}
\begin{Highlighting}[]
\FunctionTok{cat}\NormalTok{(}\StringTok{"Potential Outliers:"}\NormalTok{, }\FunctionTok{boxplot.stats}\NormalTok{(iris}\SpecialCharTok{$}\NormalTok{Sepal.Width)}\SpecialCharTok{$}\NormalTok{out, }\StringTok{"}\SpecialCharTok{\textbackslash{}n}\StringTok{"}\NormalTok{)}
\end{Highlighting}
\end{Shaded}

\begin{verbatim}
## Potential Outliers: 4.4 4.1 4.2 2
\end{verbatim}

\begin{Shaded}
\begin{Highlighting}[]
\CommentTok{\# Box plot grouped by a categorical variable (e.g., Species)}
\FunctionTok{boxplot}\NormalTok{(Sepal.Width }\SpecialCharTok{\textasciitilde{}}\NormalTok{ Species, }\AttributeTok{data =}\NormalTok{ iris, }\AttributeTok{main =} \StringTok{"Box Plot of Sepal Width by Species"}\NormalTok{, }
        \AttributeTok{xlab =} \StringTok{"Species"}\NormalTok{, }\AttributeTok{ylab =} \StringTok{"Sepal Width"}\NormalTok{, }\AttributeTok{col =} \FunctionTok{c}\NormalTok{(}\StringTok{"red"}\NormalTok{, }\StringTok{"blue"}\NormalTok{, }\StringTok{"green"}\NormalTok{))}
\end{Highlighting}
\end{Shaded}

\includegraphics{meetin1_files/figure-latex/box_plots-2.pdf}

\begin{Shaded}
\begin{Highlighting}[]
\CommentTok{\# Add observations about median, quartiles, and outliers}
\CommentTok{\# Calculate and display statistics for each species}
\FunctionTok{by}\NormalTok{(iris}\SpecialCharTok{$}\NormalTok{Sepal.Width, iris}\SpecialCharTok{$}\NormalTok{Species, }\ControlFlowTok{function}\NormalTok{(x) \{}
    \FunctionTok{cat}\NormalTok{(}\StringTok{"Species:"}\NormalTok{, }\FunctionTok{unique}\NormalTok{(iris}\SpecialCharTok{$}\NormalTok{Species[iris}\SpecialCharTok{$}\NormalTok{Sepal.Width }\SpecialCharTok{\%in\%}\NormalTok{ x]), }\StringTok{"}\SpecialCharTok{\textbackslash{}n}\StringTok{"}\NormalTok{)}
    \FunctionTok{cat}\NormalTok{(}\StringTok{"  Median:"}\NormalTok{, }\FunctionTok{median}\NormalTok{(x), }\StringTok{"}\SpecialCharTok{\textbackslash{}n}\StringTok{"}\NormalTok{)}
    \FunctionTok{cat}\NormalTok{(}\StringTok{"  1st Quartile:"}\NormalTok{, }\FunctionTok{quantile}\NormalTok{(x, }\FloatTok{0.25}\NormalTok{), }\StringTok{"}\SpecialCharTok{\textbackslash{}n}\StringTok{"}\NormalTok{)}
    \FunctionTok{cat}\NormalTok{(}\StringTok{"  3rd Quartile:"}\NormalTok{, }\FunctionTok{quantile}\NormalTok{(x, }\FloatTok{0.75}\NormalTok{), }\StringTok{"}\SpecialCharTok{\textbackslash{}n}\StringTok{"}\NormalTok{)}
    \FunctionTok{cat}\NormalTok{(}\StringTok{"  Potential Outliers:"}\NormalTok{, }\FunctionTok{boxplot.stats}\NormalTok{(x)}\SpecialCharTok{$}\NormalTok{out, }\StringTok{"}\SpecialCharTok{\textbackslash{}n\textbackslash{}n}\StringTok{"}\NormalTok{)}
\NormalTok{\})}
\end{Highlighting}
\end{Shaded}

\begin{verbatim}
## Species: 1 2 3 
##   Median: 3.4 
##   1st Quartile: 3.2 
##   3rd Quartile: 3.675 
##   Potential Outliers: 2.3 
## 
## Species: 1 2 3 
##   Median: 2.8 
##   1st Quartile: 2.525 
##   3rd Quartile: 3 
##   Potential Outliers:  
## 
## Species: 1 2 3 
##   Median: 3 
##   1st Quartile: 2.8 
##   3rd Quartile: 3.175 
##   Potential Outliers:
\end{verbatim}

\begin{verbatim}
## iris$Species: setosa
## NULL
## ------------------------------------------------------------ 
## iris$Species: versicolor
## NULL
## ------------------------------------------------------------ 
## iris$Species: virginica
## NULL
\end{verbatim}

\begin{Shaded}
\begin{Highlighting}[]
\CommentTok{\# Scatter plot for two numeric variables (e.g., Sepal.Length and Sepal.Width)}
\FunctionTok{plot}\NormalTok{(iris}\SpecialCharTok{$}\NormalTok{Sepal.Length, iris}\SpecialCharTok{$}\NormalTok{Sepal.Width, }\AttributeTok{main =} \StringTok{"Scatter Plot of Sepal Dimensions"}\NormalTok{, }
     \AttributeTok{xlab =} \StringTok{"Sepal Length"}\NormalTok{, }\AttributeTok{ylab =} \StringTok{"Sepal Width"}\NormalTok{, }\AttributeTok{col =}\NormalTok{ iris}\SpecialCharTok{$}\NormalTok{Species, }\AttributeTok{pch =} \DecValTok{19}\NormalTok{)}



\CommentTok{\# Add a legend}
\FunctionTok{legend}\NormalTok{(}\StringTok{"topright"}\NormalTok{, }\AttributeTok{legend =} \FunctionTok{levels}\NormalTok{(iris}\SpecialCharTok{$}\NormalTok{Species), }\AttributeTok{col =} \DecValTok{1}\SpecialCharTok{:}\DecValTok{3}\NormalTok{, }\AttributeTok{pch =} \DecValTok{19}\NormalTok{)}
\end{Highlighting}
\end{Shaded}

\includegraphics{meetin1_files/figure-latex/scatter_plots-1.pdf}

\begin{Shaded}
\begin{Highlighting}[]
\CommentTok{\# Grouped statistics (e.g., mean Sepal.Length by Species)}
\FunctionTok{aggregate}\NormalTok{(Sepal.Length }\SpecialCharTok{\textasciitilde{}}\NormalTok{ Species, }\AttributeTok{data =}\NormalTok{ iris, }\AttributeTok{FUN =}\NormalTok{ mean)}
\end{Highlighting}
\end{Shaded}

\begin{verbatim}
##      Species Sepal.Length
## 1     setosa        5.006
## 2 versicolor        5.936
## 3  virginica        6.588
\end{verbatim}

\begin{Shaded}
\begin{Highlighting}[]
\CommentTok{\# Customized scatter plot}
\FunctionTok{plot}\NormalTok{(iris}\SpecialCharTok{$}\NormalTok{Sepal.Length, iris}\SpecialCharTok{$}\NormalTok{Sepal.Width, }\AttributeTok{main =} \StringTok{"Customized Scatter Plot of Sepal Dimensions"}\NormalTok{, }
     \AttributeTok{xlab =} \StringTok{"Sepal Length"}\NormalTok{, }\AttributeTok{ylab =} \StringTok{"Sepal Width"}\NormalTok{, }\AttributeTok{col =}\NormalTok{ iris}\SpecialCharTok{$}\NormalTok{Species, }\AttributeTok{pch =} \DecValTok{19}\NormalTok{, }\AttributeTok{cex =} \FloatTok{1.5}\NormalTok{)}
\FunctionTok{legend}\NormalTok{(}\StringTok{"topright"}\NormalTok{, }\AttributeTok{legend =} \FunctionTok{levels}\NormalTok{(iris}\SpecialCharTok{$}\NormalTok{Species), }\AttributeTok{col =} \DecValTok{1}\SpecialCharTok{:}\DecValTok{3}\NormalTok{, }\AttributeTok{pch =} \DecValTok{19}\NormalTok{, }\AttributeTok{title =} \StringTok{"Species"}\NormalTok{)}
\end{Highlighting}
\end{Shaded}

\includegraphics{meetin1_files/figure-latex/summary_and_customization-1.pdf}

\end{document}
